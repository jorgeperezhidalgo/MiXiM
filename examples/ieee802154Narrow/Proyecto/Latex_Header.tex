\documentclass[12pt,a4paper]{scrartcl}
\usepackage[ngerman]{babel}
\usepackage[latin1]{inputenc} %erlaubt �
\usepackage[T1]{fontenc}
\usepackage[3D]{movie15}
%\usepackage{m-pictex}
\usepackage{ifpdf}
	\ifpdf
		\usepackage[pdftex]{graphicx}
	\else
%		\usepackage[divps]{graphicx}
	\fi
\usepackage{dsfont}
\usepackage{a4wide,amsmath,amssymb,eso-pic}
\usepackage{setspace}
\usepackage{nccmath}
\usepackage{amssymb}
\usepackage{amsmath}
\usepackage[mathscr]{eucal}
\usepackage{makeidx}
\usepackage{graphicx, color}
\usepackage{floatflt,float}
\usepackage[activate=normal]{pdfcprot} %soll den Blocktext besser aussehen lassen
\usepackage{capt-of} %wegen Tabelle und Bilder nebeinander
\usepackage{dcolumn} %hoffentlich f\"ur Tabellenk\"opfe
\usepackage{longtable}
\usepackage{threeparttable} %f�r Fu{\ss}noten in der Tabelle
\usepackage{moreverb}
\usepackage{pdfpages}
%\usepackage{CV}

%N\"otig um bei den Gleichungen, Bildern und Tabellen eine Nummerierung mit Kapitelnummer zu erreichen
\numberwithin{equation}{section} \numberwithin{figure}{section}
\numberwithin{table}{section}
\renewcommand\theequation{\thesection.\arabic{equation}}
\renewcommand\thefigure{\thesection.\arabic{figure}}
\renewcommand\thetable{\thesection.\arabic{table}}
%%%%%%%%%%%%%%%%%%%%%%%%%%%%%%%%%%%%%%%%%%%%%%%%%%%%%%%%%%%%%%%%%%%%%%%%%%%%%%%%%%%%%%%%%%%%%%%%%%%%%%%%%%%%%%%%%%%%%%%%%%%%%%%%
\parindent0cm %verhindert Absatz einr\"ucken
%%%%%%%%%%%%%%%%%%%%%%%%%%%%%%%%%%%%%%%%%%%%%%%%%%%%%%%%%%%%%%%%%%%%%%%%%%%%%%%%%%%%%%%%%%%%%%%%%%%%%%%%%%%%%%%%%%%%%%%%%%%%%%%%
%vereinfacht die Befehle
\newcommand{\anf}[1]{"#1"} %\anf{Wort} f\"ur die "Anf\"uhrungstriche"
\usepackage[right]{eurosym}
\newcommand{\�}{\euro} %\� f\"ur das Euro Symbol
\newcommand{\un}{\underline}
\newcommand{\ergebnis}[1]{\begin{align}\underline{\underline{#1}}\end{align}} %f\"ur Ergebnis mit Nummer zentriert unterstrichen
\newcommand{\doublepage}[2]{\begin{figure}[H]\begin{center}\begin{minipage}[t]{0.5\textwidth}#1\end{minipage}\begin{minipage}[t]{0.5\textwidth}#2\end{minipage}\end{center}\end{figure}}
\newcommand{\x}[2]{&{#1}&& \text{\small$#2$}&\nonumber}
\newcommand{\xnum}[2]{&{#1}&& \text{\small$#2$}&}
\newcommand{\menge}{\mathbb}
\newcommand{\bild}[3]{\begin{figure}[H]\begin{center}\includegraphics[scale={#1}]{{#2}}#3\end{center}\end{figure}}
\newcommand{\fbild}[4]{\begin{wrapfigure}[#4]{#1\textwidth} \begin{center} \includegraphics[width=#1\textwidth]{#2}#3\end{center}\end{wrapfigure}} %\\[32ex]
%%%%%%%%%%%%%%%%%%%%%%%%%%%%%%%%%%%%%%%%%%%%%%%%%%%%%%%%%%%%%%%%%%%%%%%%%%%%%%%%%%%%%%%%%%%%%%%%%%%%%%%%%%%%%%%%%%%%%%%%%%%%%%%%
%f\"ugt ein Hintergundbild ein
%\AddToShipoutPicture{\AtPageUpperRight{{\includegraphics[scale=1]{ich1.eps}}}}
%%%%%%%%%%%%%%%%%%%%%%%%%%%%%%%%%%%%%%%%%%%%%%%%%%%%%%%%%%%%%%%%%%%%%%%%%%%%%%%%%%%%%%%%%%%%%%%%%%%%%%%%%%%%%%%%%%%%%%%%%%%%%%%%
%Hier kann der beschreibbare Bereich eingestellt werden
%\setlength{\topmargin}{0mm}
%\setlength{\headsep}{0mm}
%\setlength{\headheight}{0mm}
%\setlength{\textheight}{240mm}
%\setlength{\textheight}{297mm}
%\setlength{\oddsidemargin}{0mm}
%\setlength{\evensidemargin}{0mm}
%\setlength{\textwidth}{0mm} %Breite des Textblockes
%\setlength{\textheight}{0mm} %H\"ohe des Textblockes
%%%%%%%%%%%%%%%%%%%%%%%%%%%%%%%%%%%%%%%%%%%%%%%%%%%%%%%%%%%%%%%%%%%%%%%%%%%%%%%%%%%%%%%%%%%%%%%%%%%%%%%%%%%%%%%%%%%%%%%%%%%%%%%%
%sorgt f\"ur die Gestaltung der Kopf und Fu{\ss}zeile
\usepackage[automark]{scrpage2}
\pagestyle{scrheadings}
\ihead{\bfseries \headmark} \chead{} \ohead{\pagemark}
\ifoot{\footnotesize} \cfoot{} \ofoot{\pagemark}
\setkomafont{pagehead}{\normalfont\rmfamily\bfseries}
\setkomafont{pagefoot}{\normalfont\rmfamily}
\setkomafont{pagenumber}{\normalfont\rmfamily\bfseries}
\setheadsepline{0.4pt}
\setfootsepline{0.4pt}
\setkomafont{disposition}{\normalcolor\rmfamily\bfseries} %�berschriften einstellen
%%%%%%%%%%%%%%%%%%%%%%%%%%%%%%%%%%%%%%%%%%%%%%%%%%%%%%%%%%%%%%%%%%%%%%%%%%%%%%%%%%%%%%%%%%%%%%%%%%%%%%%%%%%%%%%%%%%%%%%%%%%%%%%%
%zum Nutzen von textinternen Links: Inhaltsverzeichnis etc. oder Internetlinks
\usepackage{hyperref}
\definecolor{grau}{rgb}{0.19,0.19,0.19}
\hypersetup{pdftitle = {titel},
						pdfsubject = {subjekt},
						pdfauthor = {Jean-Marie Birkenmaier},
						pdfkeywords={keywords}
						colorlinks  = true,
						linkcolor   = grau,
						urlcolor    = grau,
						citecolor   = grau,      
						}
%%%%%%%%%%%%%%%%%%%%%%%%%%%%%%%%%%%%%%%%%%%%%%%%%%%%%%%%%%%%%%%%%%%%%%%%%%%%%%%%%%%%%%%%%%%%%%%%%%%%%%%%%%%%%%%%%%%%%%%%%%%%%%%%
%Grundschriftart f\"ur das Dokument einstellen
%itdefault .... Italic
%scdefault .... Romancaps
%rmdefault .... Roman (standart)
%sfdefault .... Sans Serif
\renewcommand{\familydefault}{\rmdefault}
%\usepackage{helvet} %entspricht Arial
%%%%%%%%%%%%%%%%%%%%%%%%%%%%%%%%%%%%%%%%%%%%%%%%%%%%%%%%%%%%%%%%%%%%%%%%%%%%%%%%%%%%%%%%%%%%%%%%%%%%%%%%%%%%%%%%%%%%%%%%%%%%%%%%
%f\"ur Quelltext
\usepackage{listings} 
\lstset{
identifierstyle=\ttfamily\small,
keywordstyle=\ttfamily\bf\small,
numbers=left,
language= Java,
tabsize=4,
frame=single,
stepnumber=5,
breaklines=true,
}

\begin{document}

\thispagestyle{empty}

\Large

\begin{center}

\textbf{TECHNISCHE UNIVERSTIT�T DRESDEN}\\[1.0cm]

\textbf{FAKULT�T ELEKTROTECHNIK}\\
\textbf{UND INFORMATIONTECHNIK}\\[1.0cm]

\textbf{Institut f�r Nachrichtentechnik}\\[2.0cm]

\textbf{Studienarbeit}\\[4.0cm]

\end{center}

\normalsize

Thema: Statistical Characterization of RSSI for Localization\\[4.0cm]

Vorgelegt von: \href{mailto:cb07280@qmul.qc.uk}{Jean-Marie Birkenmaier}\\[0.5cm]

Betreuer: DI Jorge Juan Robles\\

Verantwortlicher Hochschullehrer: Prof. Dr.-Ing. Lehnert\\

Tag der Einreichung: .2011\\

\newpage

\thispagestyle{empty}

%\includepdf[pages=1]{Aufgabenstellung.pdf}

\newpage

\thispagestyle{empty}

\Large
\textbf{Selbstst�ndigkeitserkl�rung}\\[1.0cm]
\normalsize

Hiermit versichere ich, dass ich die vorliegende Arbeit ohne unzul�ssige Hilfe Dritter und
ohne Benutzung anderer als der angegebenen Hilfsmittel angefertigt habe; die aus fremden
Quellen direkt oder indirekt �bernommenen Gedanken sind als solche kenntlich gemacht. Bei
der Auswahl und Auswertung des Materials sowie bei der Herstellung des Manuskripts habe
ich Unterst�tzungsleistungen von folgenden Personen erhalten:\\[1.0cm]


Weitere Personen waren an der geistigen Herstellung der vorliegenden Arbeit nicht beteiligt.
Mir ist bekannt, dass die Nichteinhaltung dieser Erkl�rung zum nachtr�glichen Entzug des
Diplomabschlusses f�hren kann.\\[0.5cm]

Dresden, den 

\newpage

\thispagestyle{empty}

\onehalfspacing

\Large
\textbf{Kurzfassung}
\normalsize

\Large
\textbf{Abstract}
\normalsize

\newpage

\pagenumbering{Roman}

\tableofcontents

\newpage

\section{Abk�rzungen und Symbolzeichen}

\begin{tabbing}
$\mu$A ~~~~~~~~\= Microampere\\
RSSI \> Received Signal Strength Indicator\\
WLAN \> Wireless Local Area Network\\
$f$ \> �bertragungsfrequenz\\
$P_{\textnormal{RF}}$ \> Empfangsleistung\\
$U$ \> Spannungsversorgung\\

\end{tabbing}

\newpage

\section{Vorwort}

\newpage

\pagenumbering{arabic}

\section{Einleitung}

\newpage

\section{Hintergrund}

\newpage

\section{Hauptteil}

\newpage

\section{Zusammenfassung}

\newpage

\section{Ausblick}

\newpage

\Large
\textbf{Anlagen}
\normalsize
\addcontentsline{toc}{section}{Anlagen}

\listoffigures
\addcontentsline{toc}{subsection}{Abbildungsverzeichnis}

\listoftables
\addcontentsline{toc}{subsection}{Tabellenverzeichnis}

\end{document}