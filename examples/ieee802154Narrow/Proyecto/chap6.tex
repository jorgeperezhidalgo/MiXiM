\chapter{Conclusions and future work}
\label{chap:conclusionsandfuture}

\section{Summary and Conclusions}

In this work the design of a High Configurable Protocol based on 802.15.4 standard is proposed. This protocol appears as a combination of two
previous proposals: \ac{LPL} and \ac{OLP}. Unlike this two proposals, this protocol allows \acp{MN} to operate in different modes according to 
their necessities and the situation of the network. Four operation modes are proposed. Mode 1 allows \acp{AN} or the Computer to estimate the 
\ac{MN}'s position using the reports sent by the \acp{MN}. This reports contain \ac{RSSI} measurements. This mode is equivalent to \ac{OLP}
approach. \acp{MN} in Mode 2 estimate their position their selves using low-complexity algorithms, avoiding the overload of the network. This 
position is estimated thanks to the captured \ac{RSSI} values. Mode 3 is used within \acp{MN} with a critical battery status. To make them 
consume as low as possible, this \acp{MN} do not listen to the channel but send their own broadcast packets to be detected by the \acp{AN}. 
This mode is equivalent to \ac{LPL}. And finally, an operational Mode 4 used by the \acp{MN} which need a high position accuracy. This is 
done combining the functionality in Modes 1 and 3.

The protocol's functionality is divided into periods which will be repeated periodically. Each period is divided into seven phases. During 
the first, fourth and sixth phases (Sync Phases) all nodes in the network are synchronized thanks to broadcast packets sent by the \acp{AN}. 
This broadcast packets are also used to provide to the \acp{MN} with \ac{RSSI} samples. The synchronization phase is divided into three parts 
to avoid correlation in \ac{RSSI} values between consecutive broadcast packets. During the second phase (Report Phase), all \acp{MN} send their
reports in case they have to. \acp{MN} configured in Mode 4 send also their broadcast packets. The third phase (\ac{VIP} Phase) is reserved for the \acp{MN} in Mode
3. During this phase they do not need to compete, with \acp{MN} in other modes, for the channel, to be able to send all their broadcast packets. 
The fifth and seventh phases are reserved for communication among \acp{AN} and the Computer.

This new protocol allows, changing the \ac{MN}'s configuration, to distribute the working periods of each \ac{MN} in time. This way the network
load will be distributed avoiding periods highly loaded compared with the others.

In the results, it is proved that \acp{MN} in Mode 3 consume much less energy than the \acp{MN} operating in other modes and, how,
this is independent of the simulated configuration. This shows how listening to broadcasted packets instead of transmitting them arises
the energy consumption. When the number of \acp{MN} in Modes 3 and 4 grows, it is seen, how, the energy consumption 
for this \acp{MN} also grows and, the performance of the ComSink Phases decreases. Hence, it is proved that \ac{LPL} configuration is not enough
for all situations, specially when the number of \acp{MN} is big. On the other hand, if all \acp{MN} were configured in Modes 1 and 2, the accuracy
and the energy consumption would be affected, revealing that \ac{OLP} configuration is not enough either. It is proved then, the necessity of
a High Configurable Protocol.

From the results can also be extracted that, independently of the configuration, the ComSink Phases are not enough to put up with all generated 
traffic. It becomes essential then to make these phases' duration longer, to reduce the overall traffic in the network by configuring most of 
the \acp{MN} in Modes 1 and 2 or to have a good distribution in time to avoid traffic concentration.

In this work two different approaches for Sync Phase were also studied. In the first approach the \acp{AN} transmit their broadcast packets randomly
in time. The best random transmission case, due to maximum random time between deliveries, is searched. In the second approach the \acp{AN} transmit
their broadcast packets distributed into time slots previously calculated by the Computer. From the results can be extracted that slotted 
transmission performance is better than even the best case of random transmission. Slotted transmission was, therefore, the one used during 
the Sync Phase in the complete protocol analysis. An automatic algorithm to calculate the slot distribution is also presented in this work.


\section{Future Work}

This work presents a stable and functional framework for a High Configurable Protocol based on 802.15.4 standard. However, all possible aspects
for a full protocol are not still contemplated. The following aspects could be still improved or developed:

\begin{itemize}
  \item \textbf{Change \ac{MN}'s configuration}. It was repeatedly said that \acp{MN} must adapt to network conditions and their own necessities. 
  The need of a \ac{MN}'s configuration change during the simulation depending on this aspects is still not contemplated in this work. The
  \acp{MN} are just configured at the beginning of the simulation.

  \item \textbf{Mobile \acp{MN}}. All this study was done for the case where all \acp{MN} stay in their original places. A study when the 
  \acp{MN} move around the playground would be also interesting.

  \item \textbf{Optimal configuration}. In this work a couple of configurations were tested to extract the first conclusions about
  this protocol. When using this protocol in a real situation, a deeper study of the configurations must be done. This way an optimal configuration
  could be used in each network situation.

  \item \textbf{Real network comparison}. Whenever this protocol is build in real devices, a deep comparison between simulation and real case should
  be done, to test the simulation validity.

  \item \textbf{Network division}. When network is too big, a division of the network into different areas could be beneficial, to avoid the high
  traffic during the ComSink Phases.

  \item \textbf{Broadcast \acp{AN}}. Another interesting case to be studied is an \acp{AN} division into two groups. One group would be the 
  \acp{AN} saw during this work, but, without the broadcasting function. A second group would be formed by \acp{AN} with a high power ability 
  which will transmit only the broadcast packets at a high power. This \acp{AN} will be able to reach all the network, synchronizing it 
  easily and in less time. This way idle time in \acp{MN} would be reduced and, for big networks, the synchronization problems due to long paths 
  and, the high routing traffic, would be avoided.

  \item \textbf{Reuse erased packets}. In this work packets are erased under many circumstances. In a future work these packets could be recovered
  or reused somehow not to lose them.

  \item \textbf{Full Network Layer}. To reduce the complexity of this project, the network layer routing was fixed, simulating always with the same
  \ac{AN}'s distribution. In a future, the Network Layer should calculate the routing automatically depending on the \ac{AN}'s position.

  \item \textbf{Aggregation and Segmentation}. In this work, \acp{AN} send a report for each \ac{MN}'s communication. To reduce traffic during
  ComSink Phases, it could be positive aggregating communications from different \acp{MN} in just one report. For this, aggregation must be added 
  to the protocol. When aggregated packets become too big, they must be segmented into smaller packets. This two capabilities should be implemented
  in the Transport Layer which does nothing in this work.

  \item \textbf{Real information in packets}. As this work's aim is to develop a working framework to test the protocol. All packet's 
  information is not real, being the only important aspect of a packet, its size. For a more realistic behavior, packets should carry 
  real information and, its handling should depend on this information.

  \item \textbf{ComSink Phases study}. It was seen how this phases are the most critical in this protocol. A deeper study of this phases
  should be done. An analysis of the advantages and disadvantages of, joining them in one bigger ComSink, taking out the up-links and down-links 
  restriction or further studies, should be done.

  \item \textbf{Queue priorities}. This work assumes all message's priority the same, putting them in a \ac{FIFO} queue. But as all messages have
  not the same importance, a priority queue should be developed. The messages should also have a life time, as some messages would be occupying 
  the queues even when they are not important anymore.

\end{itemize}
