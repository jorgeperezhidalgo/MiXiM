\appendix
\chapter{Source code first contact}
\label{chap:installation}

This appendix will give a basic explanation about the control version tool used during this work. It will also introduce how to deal with 
the contents in the \ac{DVD} for the first time. All the steps will be explained for the Linux case, more precisely for Ubuntu 10.04.

\section{Git Introduction}

All backups during this work were done with a version control program called GIT \cite{git}. It is recommended to read a good tutorial to get
to work with it \cite{manualgit}.

In GIT, there are two important files:

\begin{itemize}
  \item \textbf{.gitconfig} - This file is located on the home directory and its content defines the author, some shortcuts and other configurations.
  The file used in this work is like follows.
  \begin{verbatim}
    [user]
      name = Jorge Perez
      email = jorgeperezhidalgo@gmail.com
    [alias]
      co = checkout
      ci = commit
      st = status
      br = branch
      hist = log --pretty=format:\"%h %ad | %s%d [%an]\" --graph --date=short
      type = cat-file -t
      dump = cat-file -p
    [merge]
      tool = vimdiff
  \end{verbatim}

  Before continuing reading, it is recommended to create also this file changing the name and email because from now on, all aliases are going to
  be used. A copy of this file can be found in the root directory of the \ac{DVD}.

  \item \textbf{.gitignore} - Not all files in the working directory are important to be saved in a backup, for example, compiled files, documents, 
  etc. are not important. In this file it is defined which files are going to be saved and which not. This file should be placed in the root working
  directory, in this case the $MiXiM$ directory. To get a detailed description of this file's format, please refer to \cite{manualgit}. A copy of 
  this file can be found in $MiXiM$ directory in the \ac{DVD}.
  
\end{itemize}


\section{First contact}

First thing to do is installing \ac{OMNeT++} 4.0 \cite{OMNeT}. It can be easily done following \cite{installationomnet}. To follow this guide, 
\ac{OMNeT++} 4.0 must be installed in the home directory.

After that, copy the $MiXiM$ directory from the \ac{DVD} to the $~/omnetpp-4.1/samples/$ directory and, open the \ac{OMNeT++} 4.0.

Once the program is opened, the MiXiM project must be imported. For that click on ``File'' and than ``Import''. In the new window select 
``Existing Projects into Workspace'' from the ``General'' menu. Assure ``Select root directory'' is selected and browse the 
$~/omnetpp-4.1/samples/MiXiM/$ directory. Than click on ``Finish''. Than the new project must be compiled, that is done by pressing ``Ctrl. + B''.
To run the last version, first a configuration should be done, this can be done following \cite{manualomnet}.

To see a history of all the versions in the project, from a terminal window and inside $~/omnetpp-4.1/samples/MiXiM/$ directory, $git hist$
must be execute, obtaining the following output.

\footnotesize{
\begin{verbatim}
~/omnetpp-4.1/samples/MiXiM$ git hist
* e821eb8 2011-08-19 | Correcting document (HEAD, origin/master, origin/HEAD, master) [Jorg...
* fe8ce36 2011-08-17 | Chapters 1 to 5 ready. Chapters 1 to 3 reviewed with Caro [Jorge Perez]
* 0f6887c 2011-08-15 | Correcting chapters and finishing chapter 4 in document [Jorge Perez]
* af6a4d3 2011-08-11 | Completing document chapter 4 [Jorge Perez]
* 682758c 2011-08-10 | Revised all graphics and finished chapter 5 in document [Jorge Perez]
* d12d184 2011-08-08 | Started document chapter 5, done some new graphics [Jorge Perez]
* 907b88b 2011-08-08 | Added another slots number graphic [Jorge Perez]
* 8baff00 2011-08-07 | Starting document chapter 4 [Jorge Perez]
* f41f67d 2011-08-06 | Ended document chapter 3 [Jorge Perez]
* 63d3c92 2011-08-05 | Continuing document chapter 3 [Jorge Perez]
* e163b2f 2011-08-04 | First contact with final simulation [Jorge Perez]
* c37d1b1 2011-08-02 | Document Chapter 3 start [Jorge Perez]
* be45e73 2011-07-27 | Added different packet sizes and different network configurations [J...
* a56a4b5 2011-07-26 | Document Chapter 1 and 2 ready [Jorge Perez]
* db46eca 2011-07-23 | LaTeX document structure done [Jorge Perez]
* c201800 2011-07-21 | Corrected error: going to Rx when a transmission ready, in App layer...
* 790553c 2011-07-13 | First steps writing the project with LaTeX [Jorge Perez]
* 8046976 2011-06-06 | Sleeping the Nodes and waking them up [Jorge Perez]
* aa8459d 2011-06-01 | Restructuring the configuration phases to reduce the number of event...
* 1cb3f90 2011-05-19 | Solving some problems when having a Broadcast MN type [Jorge Perez]
* 796160b 2011-05-04 | Mobile Nodes Types and Behaviour (frequencies, active and inactive f...
* 3bf9dfb 2011-05-02 | Slotted & Random Modes comparison graphics saved and commented [Jorg...
* 7cf1f71 2011-04-26 | RSSI info forwarded to App layer to start with phase organization [J...
* e912ccb 2011-04-18 | Corrected some general misstakes and ROUTING WORKING [Jorge Perez]
* 467d069 2011-04-13 | Added automatic grid location for Anchors [Jorge Perez]
* cfd79e0 2011-04-12 | Comparison between sloted phase 1 and random phase 1 [Jorge Perez]
* 88d0229 2011-04-11 | Check the total number of slots in sync phase when changing the dens...
* 2cf324f 2011-04-11 | Added Net and Trans Layer for Nodes and Computer [Jorge Perez]
* 6fa2e53 2011-04-07 | Added Net and Trans Layer to Anchor and deleted not useful files fro...
* f2d8e00 2011-04-04 | Added computer entity and anchor density distribution in map [Jorge ...
* f678d11 2011-02-22 | Added random transmision in phase 1 with results in Matlab [Jorge Pe...
* 04db6aa 2011-02-03 | Anchors transmit in every slot and model adapted to all phases [Jorg...
* 54dd74f 2011-01-25 | Added the slot division and assigned one or more slots to every Anch...
* 871f78d 2011-01-20 | Place anchor and mobile nodes randomly in the map with minimum separ...
* d713063 2011-01-20 | First Commit [Jorge Perez]
\end{verbatim}
}

It can be observed in order how each version has a reference number, the modification date, a description and the name of the person who modified
it. The ``HEAD'' word indicates in which 

