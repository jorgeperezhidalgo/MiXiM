\chapter{INTRODUCCIÓN}

Actualmente existe un proyecto llamado AUTOPIA, desarrollado por el Instituto de Automática Industrial del Consejo Superior de Investigaciones Cien\-tí\-fi\-cas (CSIC)\cite{web:autopia}. Este proyecto consiste en el diseño de un sistema de control automático para conseguir que los coches sean capaces de circular sin tripulación, es decir, que sean capaces de mantenerse en la carretera y controlar su velocidad en caso de acercamiento de otro vehículo, realizar adelantamientos~\dots \ ,sin necesidad de control humano. Estos vehículos utilizan una combinación de sistema DGPS (Diferential GPS) y EGNOS (European Geostationary Navigation Overlay Service) que les da una precisión de posicionamiento muy elevada, alrededor de dos centímetros de error, en conjunto con escáner laser y sistemas de control de velocidad y tracción electrónicos.\nota{poner egnos en la presentación}\par
Sin embargo no están diseñados para buscar un camino en un entorno completamente desconocido, que es la funcionalidad más relevante de nuestro proyecto. También hay que tener en cuenta que el sistema de control utilizado en el proyecto AUTOPIA está compuesto por un ordenador personal y un sistema robótico para controlar el vehículo. En cambio, en nuestro caso, el ``cerebro'' es un microcontrolador fabricado por Microchip, un PIC de la familia 18F que, aunque es de gama alta, no tiene la capacidad de proceso ni almacenamiento que posee un PC. Podemos ver en la figura \ref{comp-sistemas}\ la diferencia de volumen de ambos sistemas de control.\par
\dosfig{fotos/torrecontrol.eps}{fotos/controlautopia.eps}{Comparación de sistemas}{comp-sistemas}
\par
En cuanto al aspecto funcional, el guía robótico es más parecido al perro lazarillo que a los autómatas del proyecto AUTOPIA. Por tanto se hace necesaria una comparativa entre el can y el robot.\par
El perro lazarillo es adiestrado para encontrar un camino seguro para su dueño. Le informa de obstáculos como escalones o rampas a traves de la correa. Tiene en cuenta la altura del dueño a la hora de pasar por debajo de toldos, andamios, arboles\dots \ Y siempre busca el mejor camino, el más seguro.\par
Por otro lado el can tiene necesidades, como todo ser vivo, de comida, aseo y ejercicio, aunque, como bien sabemos todos los que tenemos o hemos tenido animales de compañía en casa, esas necesidades no son comparables al cariño que recibimos de ellos.\par
En cambio, el robot tiene otras capacidades que también son dignas de mención. El guia robótico, que en adelante llamaremos \bott, no solo es capaz de esquivar obstaculos, no chocar con las paredes y esperar en los semáforos en rojo, sino que puede memorizar rutas, por ejemplo descargadas de Internet. Gracias a ésto puede guiar al usuario por lugares que éste desconoce, y por las cuales el robot no ha pasado nunca, sin ningún problema. Cosa que un perro es completamente incapaz de hacer. También se podría incorporar un altavoz con una grabación o sintetización de voz, convirtiendose el \bott, de este modo, en un guia turístico, capaz de describir los lugares por los que pasa o bien dar alertas acusticas del estilo de ``semáforo rojo'' o ``abra la puerta'' y un largo etcétera. Un caso en el que se podría aplicar esta funcionalidad de guía turístico es por ejemplo en el Forum, ayudando a personas con deficiencias visuales a llegar hasta los distintos edificios, o bien en las Olimpiadas 2012.\par
Otra capacidad característica de cualquier sistema electrónico programable es precisamente eso, se programa en cuestión de segundos con toda la información necesaria para funcionar correctamente, sin necesidad de adiestramiento, aunque también puede ser programado para seguir aprendiendo a base de adiestramiento, por ejemplo aprender rutas recorriéndolas mediante control manual, con lo que un \bott \ puede ser construido y puesto en funcionamiento en cuestión de días.\par
En cuanto al mantenimiento del \bott, éste se reduce a carga de baterías y actualizaciones de software/firmware para agregar o quitar algún accesorio, o para actualizar las rutas. Esto último nos ofrece la posibilidad de cambiar de ciudad, ir de viaje sin peligro de perdernos ni necesidad de memorizar mapas o llevarlos encima.\par
Todo ello conlleva una gran ventaja del \bott \ respecto al can en el aspecto funcional. Ahora vamos a analizarlo desde el punto de vista económico que, aunque de menor importancia que el funcional, ya que se trata de una aplicación muy específica, también hay que tenerlo en cuenta.\par
Economicamente hablando el can es más costoso que el \bott. El can vacunado, adiestrado y con al menos un par de años de edad, necesarios para completar el adiestramiento, puede alcanzar facilmente los \EUR{10.000,00}, además, como ya hemos mencionado antes, el can necesita comer, ir al veterinario, a la peluquería\dots \ en general los gastos pueden superar los \EUR{1.000,00} anuales\cite{web:perro}. El \bott, a su vez, también tiene un coste inicial que puede rondar los \EUR{3.000,00}, y los gastos de mantenimiento se reducen considerablemente.\par
Ahora que ha quedado patente la ventaja, tanto funcional como económica, del \bott\ sobre el can, ya es hora de ponernos manos a la obra.\par
El objetivo final de este proyecto es construir un robot autómata capaz de alcanzar un punto destino facilitado en coordenadas UTM (Universal Transverse Mercator) en un entorno dinámico. Es decir, un entorno en el cual puede haber obstáculos o no, estos obstáculos pueden estar en movimiento o quietos, pueden ser grandes o pequeños, altos o bajos, en definitiva desconocidos, tanto en tamaño, posición y movilidad.\par
La única forma de saber la posición en todo momento en un entorno abierto y desconocido es, por ahora, mediante GPS. Por tanto, para lograr nuestro objetivo, utilizaremos un módulo GPS OEM (Original Equipment Manufacturer) de Trimble integrado en el sistema. La mayor desventaja del sistema GPS es que el error de posicionamiento de los receptores disponibles para este proyecto puede ser de varios metros. Esto hace posible que, estando en un lado de la calle, el GPS nos indique que estamos al otro lado y al intentar cruzar nos topemos con un edificio. Para solucionar este problema de precisión tendremos que implementar un receptor DGPS-EGNOS, o bien, esperar a la puesta en funcionamiento del sitema europeo de posicionamiento por satélite, llamado GALILEO\cite{web:galileo}.\par
El sitema GALILEO está previsto que entre en funcionamiento para el año 2008. Constará de 30 satélites, los cuales aparte de la información de posicionamiento, proveeran a los receptores de información meteorológica, información de tráfico y accidentes, etc.\par
El proyecto GALILEO está también incluido en el sistema EGNOS que permite que la informacion de corrección de errores diferencial sea enviada directamente por los satélites GEO (Geoestacionarios, utilizados para telecomunicación). De esta forma, solo es neceasario un receptor compatible con EGNOS. Se espera que a partir del lanzamiento a nivel operativo de GALILEO se produzca un cambio significativo en la navegación por satélite, ya que a partir de entonces habrá más de 70 satélites, entre los distintos sistemas, emitiendo señales de posicionamiento. Para mayores referencias técnicas hemos incluido un anexo dedicado al sistema GALILEO.\par
Ahora que ya hemos enmarcado el \bott\ en la actualidad del guiado, tanto de personas con deficiencias visuales como de autómatas móviles, y habiendo establecido el objetivo del proyecto, describiremos la estructura y contenido de esta memoria.
\par En el capítulo segundo se describe el problema de enrutamiento y las posibles soluciones, así como la solución adoptada para nuestro \bott.
\par El capítulo tercero versará sobre las herramientas utilizadas.
\par El capítulo cuarto abarcará el diseño, analisis y posterior rediseño del hardware.
\par En el capítulo quinto explicaremos detalladamente los algoritmos implementados en el \bott.
\par Ya en el capítulo sexto encontrarémos el informe sobre la pruebas realizadas y los resultados obtenidos en ellas.
\par El capítulo septimo se centrará en las conclusiones, y futuras lineas de investigación y desarrollo.
\par El capítulo octavo contiene el pliego de condiciones
\par El capítulo noveno nos muestra el presupuesto del \bott.
\par Y, por último, un anexo sobre los distintos sistemas de posicionamiento por satélite, incluido el sistema Galileo que sustituirá al sistema GPS en Europa.