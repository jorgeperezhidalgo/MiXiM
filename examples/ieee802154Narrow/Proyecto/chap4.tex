\chapter{Protocol Implementation}
\label{chap:protocolimplementation}

This chapter details the simulation environment where the presented High Configurable Protocol was developed and tested. Simulation was chosen because
both analytical and real networks studies, are difficult or expensive in such a complex protocol. During this chapter a deeper view into the protocol will
be given, as all generated code will be explained in detail.

\section{Used Tools}

To develop this project different simulators and frameworks for this simulators where considered, choosing finally \ac{OMNeT++} 4.0 \cite{OMNeT}
as simulator, and \ac{MiXiM} 2.0 \cite{MiXiM} as framework, due to its versatility, their short time to get to work with them for the first time and 
their complete development of 802.15.4 standard, specially in non-beaconed mode.

After all simulations are done, the results can be extracted with a tool called \textit{scavetool}. This data is than worked with \ac{MATLAB} 
\cite{MATLAB} to obtain all results in Chapter \ref{chap:simulationandresults}: \nameref{chap:simulationandresults}.

\subsection{\ac{OMNeT++} 4.0}

\ac{OMNeT++} 4.0 is an object oriented discrete event simulator, based in C++ \cite{cpp}. \ac{OMNeT++} consists in several modules hierarchically
connected and that communicate among them through messages. Modules relation is done through an own easy programming language called \ac{NED}.

This software allows two kinds of simulation environments, a graphic one (Tkenv) and a command line one (Cmdenv). Working with the graphic one, 
message interchange simulation can be done step by step, this mode is good to debug the code. Working with the command environment, allows to
make express simulations to obtain the final results in a quicker way, in this mode it is also possible to program some
parameter changes during the simulation or even some consecutive simulations, this is all automatically done and thus does not require user 
intervention This mode makes all process much easier when lots of iterations must be done.

Explicar la configuracion de parametros en los .ned y omnetpp.ini 

Explicar la estructura de un modulo cualquiera en omnet, initialize y los tratamientos de los mensajes y finish para las variables

\subsection{\ac{MiXiM} Framework}

\ac{MiXiM} 2.0 framework provides \ac{OMNeT++} with many new modules, among all of them, all necessary modules to work with 802.15.4 Standard, are 
provided. All this modules are build following \cite{IEEE802.15.4-2006}. 

Explicar estructura de un nodo actual y las modificaciones realizadas

\subsubsection{Mobility module modifications}

\subsubsection{\ac{MAC} Layer modifications}

\section{Sync Phase study development}

\section{Framework development}
