\chapter{Introduction}

We are inside the information era, it is even said that information means money. It is clear then, that acceding to this information
everywhere and every time is an important issue. Thanks to telecommunications advances and the need of this information, it is
possible now to be permanently connected. Technology has made also possible that devices are smaller, cheaper and more powerful
every day (\ac{PDA}, Mobile Telephones, \ac{MP3} Players \ldots). All this together has created the \ac{WPAN} concept, a way 
all this devices can be interconnected, so they can complement each other to obtain applications beyond our imagination. The 
problem in all this magic world is an old problem, energy. All this devices are supplied with batteries, what means that their 
lives are not so long until we must plug them in to continue working with them. This is the reason why making them to reduce its 
energy consumption, is a really important issue.

With the objective of creating an standard for this \ac{WPAN}, the \ac{IEEE}, created the 802.15 group, who created different 
kind of networks attending to the data rate. Among this kind of networks it is found the 802.15.4, designed for low data rates but 
also for low consume. This standard is the one this work is based in.

\section{Why \ac{WSN}?}

As a cabled technology would be unpractical for a topology where free of movement is a statement, it is clear the need of a 
wireless technology, but why did we choose 802.15.4 \ac{WSN} and not other alternatives?

One possible alternative would have been Wi-Fi. This technology is based in \ac{IEEE} 802.11 standard, and with bit rates up to
54 Mbps and much bigger ranges, would be a good alternative, but its energy consumption is much bigger than for 802.15.4 \ac{WSN}.

In the same group as \ac{WSN}, it can be found the Bluetooth (\ac{IEEE} 802.15.1), this standard has still bigger bit rates as 
802.15.4, but its energy consumption is still bigger, as it is not thought for this kind of applications. Also Bluetooth scalability
is not as good as for 802.15.4.

This reasons together with the low prices of 802.15.4 compared with the other alternatives, made this work use the \ac{IEEE} 802.15.4 
standard as the standard for building a High Configurable Protocol for localization. The idea is that using this protocol we could
get a little node working with a couple of AA batteries for many months or even some years.

\section{Applications}

\ac{WSN} applications are wide and diverse, this application areas are just some examples:

\begin{itemize}
 \item \textbf{Medicine.} As the nodes size is every time smaller, some patients with problems which should be controlled all the time,
could find this networks really useful.
 \item \textbf{Security environments.} Places with toxic substances, where there cannot be humans, and where some parameters should be measured,
or delicate places that need the presence of sensors 24/7 to check that there was no intruder. This could be done also by \ac{WSN}.
 \item \textbf{Environmental sensors.} Vast areas like forests, sees, coasts \ldots that need to be controlled in some parameters like
humidity, temperature, fire, seismic activity \ldots which are impossible to be controlled by humans, are good controlled with sensors
and at the same time they minimize the environmental impact.
 \item \textbf{Industrial sensors.} The size of the sensors makes possible to access every corner of the industrial process, and it is also
possible to control much more parameters.
 \item \textbf{Indoor positioning.} Sensor make possible to be guided inside a building we don't know, or even to people with some handicaps.
They are also good to locate determinate objects in a building.
 \item \textbf{Home automation.} Every element in our house could have a sensor inside, they could communicate among them and even notify us
about some alerts or needs from the house, they could even check how we feel to prepare the house in a specific way.
\end{itemize}

Table \ref{wsn_applications} was made from all this applications and many more. This table reflects a summary and divide and group
the applications according to the priority in energy, accuracy and emergency, but also if the involved device needs to obtain some information
from the network. Priority in energy means that we don't care much about the other parameters, we just need to save energy, the same happens 
with accuracy and with emergency, where the rest doesn't matter, the exact position or speed are needed.

\begin{table}[h]\footnotesize
\begin{center}
 \begin{tabular}{lcccc}
  \noalign{\vspace*{0.5cm}}
  & \textbf{Consume} & \textbf{Accuracy} & \textbf{Emergency} & \textbf{Do I need} \\
  & \textbf{Priority} & \textbf{Priority} & \textbf{Priority} & \textbf{some data?} \\
  \hline\hline
  \textbf{Guided System} & High & High & Very High & Yes \\
  \hline 
  \textbf{Routine inspection in mobile stations} & Very High & High & Low & No \\
  \hline
  \textbf{Objects localization with accuracy} & High & High & Low & No \\
  \hline
  \textbf{Get some data about my position} & High & High & Low & Yes \\
  \hline
  \textbf{Objects localization with emergency} & High & Low & High & No \\
  \hline
  \textbf{Big objects with battery localization} & Very High & Low & Low & No \\
  \hline
  \textbf{Routine inspection in fixed stations} & Very High & Very Low & Low & No \\
  \hline
  \textbf{Get position with low energy consumption} & Very High & Low & Low & Yes \\
  \hline
  \textbf{Emergency measurement in mobile stations} & Low & High & Very High & No \\
  \hline
  \textbf{Emergency measurement in fixed stations} & Low & Very Low & Very High & No \\
  \hline
  \textbf{Plugged in objects localization} & Very Low & Low & Low & No \\
  \hline
  \textbf{High accuracy localization at any price} & Very Low & Very High & Very Low & No \\
  \hline
  \end{tabular}
 \caption{Summed and grouped \ac{WSN} applications}
 \label{wsn_applications}
\end{center}
\end{table}


\section{Challenges and Objectives}

\section{Document structure}
